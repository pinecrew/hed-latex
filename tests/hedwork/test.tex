\documentclass{hedwork}
\usepackage[russian]{babel}
\usepackage[utf8]{inputenc}
\usepackage{graphicx}
\usepackage[vectors,environments,complex,root]{hedmaths}
\usepackage[nuclear]{hedphysics}

\faculty{Факультет электроники и вычислительной техники}
\department{<<Физика>>}
\subject{Электродинамика}
% \semnum{}
% \variant{}
\student[m]{Иванов И. И.}
\teacher[f]{Петрова П. П.}

\begin{document}
\maketitle
\begin{abstract}
    Это \underline{аннотация} или \emph{реферат}. Вроде поведение с
    \textbf{титульником} и без него \textit{отличается}.
\end{abstract}
\tableofcontents
\section{Секция}
Идейные\footnote{пример сноски; текст вообще ни о чём, лучше даже не
вчитываться} соображения высшего порядка, а также реализация намеченных плановых заданий требуют определения и уточнения модели развития. Товарищи! консультация с широким активом требуют от нас анализа позиций, занимаемых участниками в отношении поставленных задач. Значимость этих проблем настолько очевидна, что постоянный количественный рост и сфера нашей активности позволяет выполнять важные задания по разработке позиций, занимаемых участниками в отношении поставленных задач. Повседневная практика показывает, что укрепление и развитие структуры представляет собой интересный эксперимент проверки системы обучения кадров, соответствует насущным потребностям. Не следует, однако забывать, что сложившаяся структура организации требуют определения и уточнения направлений прогрессивного развития. Разнообразный и богатый опыт постоянный количественный рост и сфера нашей активности представляет собой интересный эксперимент проверки форм развития.
\begin{figure}[t]
    \center
    \includegraphics[width=.47\textwidth]{image1.jpg}
    \caption[Результаты исследования (сон)]{Результаты исследования}
\end{figure}
\section{Ещё секция про эпическое интегрирование дифференциальных
уравнений}
\subsection{Эпическое}
\subsubsection{Интегрирование}
Не следует, однако забывать, что дальнейшее развитие различных форм деятельности требуют определения и уточнения соответствующий условий активизации. Не следует, однако забывать, что постоянный количественный рост и сфера нашей активности способствует подготовки и реализации форм развития. Товарищи! постоянный количественный рост и сфера нашей активности способствует подготовки и реализации системы обучения кадров, соответствует насущным потребностям. Равным образом дальнейшее развитие различных форм деятельности требуют от нас анализа новых предложений.

С другой стороны начало повседневной работы по формированию позиции позволяет выполнять важные задания по разработке соответствующий условий активизации. Равным образом консультация с широким активом влечет за собой процесс внедрения и модернизации новых предложений. Значимость этих проблем настолько очевидна, что реализация намеченных плановых заданий способствует подготовки и реализации систем массового участия. С другой стороны укрепление и развитие структуры позволяет выполнять важные задания по разработке модели развития.
\begin{figure}[ht]
    \center
    \includegraphics[width=.47\textwidth]{image2.jpg}
    \caption[Результаты исследования (поезд)]{Результаты исследования}
\end{figure}

\section{Немного математики}
Лапласиан в сферических координатах:
\begin{equation}
    \Delta\psi = \frac{1}{r^2}\frac{\partial}{\partial r}
    \left( r^2 \frac{\partial \psi}{\partial r} \right) +
    \frac{1}{r^2\sin\theta}\frac{\partial}{\partial\theta}
    \left(\sin\theta\frac{\partial\psi}{\partial\theta}\right) +
    \frac{1}{r^2\sin^2\theta}\frac{\partial^2\psi}{\partial^2\phi}.
    \label{eq:laplas}
\end{equation}
Используя формулу (\ref{eq:laplas}), выпишем вид решения уравнения лапласа для
азимутально-симметричного случая:
\begin{equation}
    \psi(r,\theta) = \sum_{l=0}^\infty\left(A_lr^l + B_lr^{-(l+1)}\right)\cdot
    P_l(\cos\theta),
\end{equation}
где \(P_l(\cos\theta)\) -- многочлен Лежандра.

А теперь проверка окружений:
\begin{description}
    \item[Description] Это окружение description...
    \item[Определение] Это список определений...
    \item[Вопрос] А оно нам нужно?
\end{description}

Списки:
\begin{itemize}
    \item один
    \begin{itemize}
         \item один один
         \begin{itemize}
             \item один один один
         \end{itemize}
         \item один два
     \end{itemize} 
     \item два
\end{itemize}
\begin{enumerate}
    \item один
    \begin{enumerate}
         \item один один
         \begin{enumerate}
             \item один один один
         \end{enumerate}
         \item один два
     \end{enumerate} 
     \item два
\end{enumerate}
    

\section{Примеры использования пакета hedmaths}
    
\begin{solution}
    Допустим, что \( \nucleus{1}{2}{L} - \nucleus{2}{1}{M} = 1. \)

    Тогда
    \[
        \sin^2\xi + \cos^2\xi = 
        \nucleus{1}{2}{L} - \nucleus{2}{1}{M}.
    \]
    Откуда с помощью простых преобразований найдем \( \xi \).
\end{solution}

\[
    \dder{}{t}\left( \pder{x}{\kappa} \right) =
    \ppder{\divergence{\vec{r}}}{\phi}\cdot\rotor{\vec\Gamma} +
    \frac{\Re{x}}{\Im{\psi^3}} - \average{\pcder{x}{\alpha}{\beta}},
\]
где \( \alpha = \sqrt[3]{\beta + \dfrac{2x}{\kappa}} \).

\begin{comment}
    При использовании пакета опции {\tt root} пакета {\tt{hedmaths}} у корня
    появится закрывающая черта:
    \[
       \Delta\sigma = \sqrt{\left( \pder{\sigma}{J}\Delta J \right)^2 +
        \left( \pder{\sigma}{U}\Delta U \right)^2 +
        \left( \pder{\sigma}{T}\Delta T \right)^2 +
        \left( \pder{\sigma}{T_0}\Delta T_0 \right)^2}.
    \]
\end{comment}

\begin{table}[ht]
    \centering
    \caption{Пример таблицы с использованием центрированного столбца
        фиксированного размера}
    \begin{tabular}{|C{.2}|C{.2}|C{.17}|C{.18}|C{.15}|}
        \hline
        Объект & Вид & Контрольный уровень & Прибор & Результат \\ \hline
        Рабочее место у манипулятора в операторной
        & Измерение мощности дозы
        & 1 мкЗв/час & ДРГ-01Т1 МКС-01Р-01
        & 0,08\( \pm \)0,04 мкЗв/час \\ \hline
        Захваты манипулятора, поверхности подвижного стола
        & Контроль р/а загрязнённости методом мазков
        & Отсутствие снимаемых загрязнений
        & МКС-01Р-01 & отсутствует \\ \hline
    \end{tabular}
\end{table}

\listoffigures
\end{document}
